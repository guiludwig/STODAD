% pandoc -s resumo_site_ENG.tex -o ../resumo_site_ENG.md
\documentclass[11pt]{article}
\usepackage{natbib}
\usepackage{dsfont} 
\usepackage{amsfonts}
\usepackage{amssymb}
\usepackage{amsmath}
\usepackage{bbm}
\usepackage{graphicx}
\usepackage[normalem]{ulem} 
\usepackage{rotating} 
\usepackage{color} 
\usepackage{pdfpages}	
\setlength{\textwidth}{140mm}  
\setlength{\textheight}{220mm}    

\begin{document}
\vspace{1cm}
\thispagestyle{empty}

\begin{center}

{\huge


{\sf Time Series, Wavelets and High Dimensional Data Analysis}

}

\vspace{1cm}

Project Manager: Pedro Alberto Morettin

\vspace{1cm}

Institution: Instituto de Matem\'atica e Estat\'\i stica, Universidade de S\~ao Paulo

\vspace{.5cm}

Thematic Project FAPESP 2018/04654-9

\vspace*{1cm}

{\sc Summary}

\end{center}

This project is a natural follow-up to the thematic project  {\sc 13/00506-1} entitled {\bf S\'eries Temporais, Ondaletas e An\'alise de Dados Funcionais}, from 01/07/2013 to 30/06/2018. The methodologies have potential and effective application to Medicine, Biology, Physics, Chemistry, Finance, Engineering, etc. These methods should solve theoretical and applied problems on the following topics: (1) Generalizations of ARMA models; (2) Wavelets; (3) Quasi $U$-statistics; (4) Outliers in time series; (5) Financial Volatilty, including high frequency data; (6) Functional data analysis; (7) High dimensional data, with emphasis in time series, spatial data, financial series, satellite images, genetics, DNA sequences, microarrays and MRI.

Results will be published in international journals with selective editorial policies and shall be presented in scientific conferences. Forming human resources is one of our main concerns. For this effect we will supervise pos-docs, undergraduate research assistantships, master theses and PhD dissertations. Seminars will be held regularly, in which the presentation of results, exchange of ideas, and where prospective talents may get acquainted with these research areas. In order to bring a serious push in this area in Brazil we will held workshops, which will reunite every year international and Brazilian guest speakers, the project team, advanced undergraduate and graduate students. State of the art and open problems will be discussed, and new and/or established scientific collaborations will happen. 

\end{document}
