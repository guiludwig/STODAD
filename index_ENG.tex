\documentclass[]{article}
\usepackage{lmodern}
\usepackage{amssymb,amsmath}
\usepackage{ifxetex,ifluatex}
\usepackage{fixltx2e} % provides \textsubscript
\ifnum 0\ifxetex 1\fi\ifluatex 1\fi=0 % if pdftex
  \usepackage[T1]{fontenc}
  \usepackage[utf8]{inputenc}
\else % if luatex or xelatex
  \ifxetex
    \usepackage{mathspec}
  \else
    \usepackage{fontspec}
  \fi
  \defaultfontfeatures{Ligatures=TeX,Scale=MatchLowercase}
\fi
% use upquote if available, for straight quotes in verbatim environments
\IfFileExists{upquote.sty}{\usepackage{upquote}}{}
% use microtype if available
\IfFileExists{microtype.sty}{%
\usepackage[]{microtype}
\UseMicrotypeSet[protrusion]{basicmath} % disable protrusion for tt fonts
}{}
\PassOptionsToPackage{hyphens}{url} % url is loaded by hyperref
\usepackage[unicode=true]{hyperref}
\hypersetup{
            pdfborder={0 0 0},
            breaklinks=true}
\urlstyle{same}  % don't use monospace font for urls
\IfFileExists{parskip.sty}{%
\usepackage{parskip}
}{% else
\setlength{\parindent}{0pt}
\setlength{\parskip}{6pt plus 2pt minus 1pt}
}
\setlength{\emergencystretch}{3em}  % prevent overfull lines
\providecommand{\tightlist}{%
  \setlength{\itemsep}{0pt}\setlength{\parskip}{0pt}}
\setcounter{secnumdepth}{0}
% Redefines (sub)paragraphs to behave more like sections
\ifx\paragraph\undefined\else
\let\oldparagraph\paragraph
\renewcommand{\paragraph}[1]{\oldparagraph{#1}\mbox{}}
\fi
\ifx\subparagraph\undefined\else
\let\oldsubparagraph\subparagraph
\renewcommand{\subparagraph}[1]{\oldsubparagraph{#1}\mbox{}}
\fi

% set default figure placement to htbp
\makeatletter
\def\fps@figure{htbp}
\makeatother


\date{}

\begin{document}

\textbf{Project Manager:} Pedro Alberto Morettin

\textbf{Institution:} Instituto de Matemática e Estatística,
Universidade de São Paulo

Thematic Project FAPESP 2018/04654-9

\textbf{News:} 1st Workshop in Time Series, Wavelets and High
Dimensional Data at the Universty of Campinas, August 29--30, 2019. More
info \href{workshop1.md}{here}

\subsection{Introduction}\label{introduction}

This project is a natural follow-up to the thematic project
\emph{2013/00506-1} entitled \textbf{Séries Temporais, Ondaletas e
Análise de Dados Funcionais}, from 01/07/2013 to 30/06/2018. The
methodologies have potential and effective application to Medicine,
Biology, Physics, Chemistry, Finance, Engineering, etc. These methods
should solve theoretical and applied problems on the following topics:
(1) Generalizations of ARMA models; (2) Wavelets; (3) Quasi
\emph{U}-statistics; (4) Outliers in time series; (5) Financial
Volatilty, including high frequency data; (6) Functional data analysis;
(7) High dimensional data, with emphasis in time series, spatial data,
financial series, satellite images, genetics, DNA sequences, microarrays
and MRI.

Results will be published in international journals with selective
editorial policies and shall be presented in scientific conferences.
Forming human resources is one of our main concerns. For this effect we
will supervise pos-docs, undergraduate research assistantships, master
theses and PhD dissertations. Seminars will be held regularly, in which
the presentation of results, exchange of ideas, and where prospective
talents may get acquainted with these research areas. In order to bring
a serious push in this area in Brazil we will held workshops, which will
reunite every year international and Brazilian guest speakers, the
project team, advanced undergraduate and graduate students. State of the
art and open problems will be discussed, and new and/or established
scientific collaborations will happen.

\subsection{Project Members}\label{project-members}

See \href{\%7B\%7B\%20site.people\%20\%7C\%20site.url\%7D\%7D}{People}.

\subsection{Research}\label{research}

See \href{\%7B\%7B\%20site.temas\%20\%7C\%20site.url\%7D\%7D}{Research
Themes}.

\emph{See this page in portuguese clicking \href{index_PORT.md}{here}.}

\end{document}
